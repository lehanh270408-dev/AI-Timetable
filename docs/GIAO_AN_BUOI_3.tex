\documentclass[12pt,a4paper]{article}
\usepackage[utf8]{inputenc}
\usepackage[vietnamese]{babel}
\usepackage{geometry}
\usepackage{graphicx}
\usepackage{hyperref}
\usepackage{enumitem}
\usepackage{xcolor}
\usepackage{titlesec}
\usepackage{fancyhdr}
\usepackage{booktabs}
\usepackage{array}

% Cấu hình trang
\geometry{left=2.5cm,right=2.5cm,top=2.5cm,bottom=2.5cm}
\pagestyle{fancy}
\fancyhf{}
\fancyhead[L]{\textbf{Giáo án buổi 3}}
\fancyhead[R]{\textbf{AI Gợi ý lớp học - Machine Learning}}
\fancyfoot[C]{\thepage}

% Định dạng tiêu đề
\titleformat{\section}
{\Large\bfseries\color{blue!70!black}}
{}
{0em}
{}[\titlerule]

\titleformat{\subsection}
{\large\bfseries\color{blue!50!black}}
{}
{0em}
{}

% Màu sắc
\definecolor{primaryblue}{RGB}{0,102,204}
\definecolor{secondaryblue}{RGB}{51,153,255}

\begin{document}

% Tiêu đề
\begin{center}
    {\Huge\bfseries\color{primaryblue} GIÁO ÁN BUỔI 3}\\[0.5cm]
    {\Large\bfseries AI GỢI Ý LỚP HỌC - MACHINE LEARNING}\\[0.3cm]
    {\large Hệ thống xếp thời khóa biểu thông minh}\\[0.5cm]
    \rule{0.8\textwidth}{0.4pt}\\[0.5cm]
    {\large Thời lượng: \textbf{2 giờ}}\\[0.3cm]
    {\large Đối tượng: Học sinh lớp 12}
\end{center}

\vspace{1cm}

% Thông tin chung
\section*{Thông tin chung}
\begin{tabular}{ll}
    \textbf{Tên dự án:} & Smart Timetable System (SPCN\_HaiAnh) \\
    \textbf{Buổi học:} & Buổi 3/6 \\
    \textbf{Thời lượng:} & 2 giờ (120 phút) \\
    \textbf{Đối tượng:} & Học sinh lớp 12, có kiến thức cơ bản về Python \\
    \textbf{Mục tiêu chính:} & Hiểu cách AI và Machine Learning gợi ý lớp học dựa trên sở thích người dùng
\end{tabular}

\vspace{0.5cm}

% Mục tiêu học tập
\section{Mục tiêu học tập}
Sau buổi học này, học sinh có thể:
\begin{itemize}[leftmargin=*]
    \item Hiểu khái niệm cơ bản về AI và Machine Learning thông qua ví dụ cụ thể
    \item Biết cách AI tính điểm và xếp hạng lớp học dựa trên sở thích người dùng
    \item Thực hành chạy AI và phân tích kết quả gợi ý
    \item Hiểu quá trình học của AI từ input đến output
    \item So sánh và đánh giá kết quả AI với các bộ sở thích khác nhau
\end{itemize}

\vspace{0.5cm}

% Phân bổ thời gian
\section{Phân bổ thời gian}
\begin{table}[h]
\centering
\begin{tabular}{@{}lcc@{}}
\toprule
\textbf{Phần} & \textbf{Nội dung} & \textbf{Thời lượng} \\
\midrule
Phần 1 & AI hoạt động như thế nào? & 45 phút \\
Phần 2 & Chạy AI và xem kết quả & 45 phút \\
Phần 3 & Hiểu cách AI học & 30 phút \\
\bottomrule
\end{tabular}
\end{table}

\newpage

% PHẦN 1
\section{Phần 1: AI hoạt động như thế nào?}
\textbf{Thời lượng:} 45 phút

\subsection{Mục tiêu}
Hiểu khái niệm cơ bản về AI và Machine Learning thông qua ví dụ cụ thể về gợi ý lớp học.

\subsection{Nội dung giảng dạy}

\subsubsection{1. AI là gì? (10 phút)}
\begin{itemize}[leftmargin=*]
    \item \textbf{Định nghĩa đơn giản:} AI (Trí tuệ nhân tạo) là khả năng máy tính học từ dữ liệu để đưa ra quyết định thông minh
    \item \textbf{Ví dụ thực tế:}
    \begin{itemize}
        \item Netflix gợi ý phim dựa trên lịch sử xem của bạn
        \item YouTube đề xuất video bạn có thể thích
        \item Hệ thống của chúng ta gợi ý lớp học phù hợp với sở thích của bạn
    \end{itemize}
    \item \textbf{Kết nối với bài học:} AI sẽ học từ sở thích của bạn (ngày, giờ, phòng) để gợi ý lớp học phù hợp nhất
\end{itemize}

\subsubsection{2. Cách AI gợi ý lớp học (15 phút)}
\begin{itemize}[leftmargin=*]
    \item \textbf{Ví dụ cụ thể:}
    \begin{itemize}
        \item Bạn thích học thứ 2, 4, 6 $\rightarrow$ AI sẽ ưu tiên lớp học vào những ngày đó
        \item Bạn thích học buổi sáng (7h-11h) $\rightarrow$ AI sẽ ưu tiên lớp học trong khung giờ đó
        \item Bạn thích phòng D5-401 $\rightarrow$ AI sẽ ưu tiên lớp học ở phòng đó
    \end{itemize}
    \item \textbf{Cách tính điểm:}
    \begin{itemize}
        \item Mỗi lớp được chấm điểm dựa trên sở thích của bạn
        \item Lớp nào phù hợp hơn $\rightarrow$ điểm cao hơn
        \item Điểm cao nhất $\rightarrow$ Lớp được gợi ý đầu tiên
    \end{itemize}
    \item \textbf{Ví dụ tính điểm:}
    \begin{itemize}
        \item Lớp A: Thứ 2, 7h-9h, phòng D5-401
        \item Sở thích của bạn: Thứ 2-4-6, 7h-11h, phòng D5
        \item Điểm: Thứ 2 (+1), 7h-11h (+1), phòng D5 (+1) = \textbf{3 điểm}
    \end{itemize}
\end{itemize}

\subsubsection{3. Random Forest - Thuật toán AI (10 phút)}
\begin{itemize}[leftmargin=*]
    \item \textbf{Giải thích đơn giản:} Random Forest là thuật toán AI học từ nhiều quyết định nhỏ
    \item \textbf{Ví dụ:}
    \begin{itemize}
        \item Thay vì 1 người quyết định, có nhiều "chuyên gia" cùng đánh giá
        \item Mỗi "chuyên gia" xem xét một khía cạnh khác nhau
        \item Kết quả cuối cùng là ý kiến của đa số
    \end{itemize}
    \item \textbf{Ưu điểm:} Chính xác hơn, ổn định hơn so với chỉ dùng 1 quyết định
\end{itemize}

\subsubsection{4. Thực hành (10 phút)}
\begin{enumerate}[leftmargin=*]
    \item Mở file \texttt{timetable\_user.csv}
    \item Sửa sở thích:
    \begin{itemize}
        \item \texttt{PreferredDays}: \texttt{Mon, Wed, Fri}
        \item \texttt{PreferredTimeSlots}: \texttt{07:00-11:00}
        \item \texttt{PreferredRooms}: \texttt{D5-401, C7-205}
    \end{itemize}
    \item Lưu file
    \item Chạy lệnh: \texttt{python ai\_recommender.py}
    \item Xem kết quả: \texttt{ai\_ranked\_classes.csv}
    \item Quan sát: Lớp nào được AI đề xuất cao nhất? Tại sao?
\end{enumerate}

\subsection{Tài liệu hỗ trợ}
\begin{itemize}[leftmargin=*]
    \item File: \texttt{timetable\_user.csv} - Cấu hình sở thích
    \item Script: \texttt{ai\_recommender.py} - Chạy AI gợi ý
    \item Kết quả: \texttt{ai\_ranked\_classes.csv} - Danh sách lớp được xếp hạng
\end{itemize}

\newpage

% PHẦN 2
\section{Phần 2: Chạy AI và xem kết quả}
\textbf{Thời lượng:} 45 phút

\subsection{Mục tiêu}
Thực hành chạy AI và phân tích kết quả gợi ý lớp học.

\subsection{Nội dung giảng dạy}

\subsubsection{1. Chạy script AI (10 phút)}
\begin{enumerate}[leftmargin=*]
    \item Mở terminal/command prompt
    \item Di chuyển đến thư mục dự án
    \item Chạy lệnh: \texttt{python ai\_recommender.py}
    \item Quan sát quá trình chạy:
    \begin{itemize}
        \item Đọc dữ liệu từ \texttt{timetable\_all.csv}
        \item Đọc sở thích từ \texttt{timetable\_user.csv}
        \item Huấn luyện mô hình Random Forest
        \item Tính điểm AI cho từng lớp
        \item Lưu kết quả vào \texttt{ai\_ranked\_classes.csv}
    \end{itemize}
\end{enumerate}

\subsubsection{2. Phân tích kết quả (20 phút)}
\begin{itemize}[leftmargin=*]
    \item \textbf{Mở file kết quả:} \texttt{ai\_ranked\_classes.csv}
    \item \textbf{Các cột quan trọng:}
    \begin{itemize}
        \item \texttt{ai\_score}: Điểm AI (càng cao = lớp càng phù hợp)
        \item \texttt{Mã học phần}: Mã lớp học
        \item \texttt{Tên môn}: Tên môn học
        \item \texttt{Giáo viên}: Tên giáo viên
        \item \texttt{Phòng}: Phòng học
    \end{itemize}
    \item \textbf{So sánh lớp có điểm cao vs điểm thấp:}
    \begin{itemize}
        \item Lớp điểm cao: Phù hợp với sở thích như thế nào?
        \item Lớp điểm thấp: Không phù hợp ở điểm nào?
        \item Giải thích: Tại sao AI đánh giá như vậy?
    \end{itemize}
    \item \textbf{Sắp xếp theo điểm:}
    \begin{itemize}
        \item Mở file CSV bằng Excel
        \item Sắp xếp cột \texttt{ai\_score} giảm dần
        \item Quan sát Top 10 lớp được AI đề xuất
    \end{itemize}
\end{itemize}

\subsubsection{3. Thử nghiệm với sở thích khác nhau (15 phút)}
\begin{enumerate}[leftmargin=*]
    \item \textbf{Thử nghiệm 1:} Thay đổi \texttt{PreferredDays}
    \begin{itemize}
        \item Trước: \texttt{Mon, Wed, Fri}
        \item Sau: \texttt{Tue, Thu}
        \item Chạy lại AI $\rightarrow$ Quan sát điểm AI thay đổi như thế nào?
    \end{itemize}
    \item \textbf{Thử nghiệm 2:} Thay đổi \texttt{PreferredTimeSlots}
    \begin{itemize}
        \item Trước: \texttt{07:00-11:00}
        \item Sau: \texttt{13:00-17:00}
        \item Chạy lại AI $\rightarrow$ Quan sát điểm AI thay đổi như thế nào?
    \end{itemize}
    \item \textbf{So sánh kết quả:}
    \begin{itemize}
        \item Top 5 lớp trước và sau khi thay đổi sở thích
        \item Giải thích: Tại sao điểm AI thay đổi?
    \end{itemize}
\end{enumerate}

\subsection{Bài tập thực hành}
\begin{enumerate}[leftmargin=*]
    \item Chạy AI với 3 bộ sở thích khác nhau
    \item Ghi lại Top 5 lớp được AI đề xuất cho mỗi bộ sở thích
    \item So sánh và giải thích sự khác biệt
    \item Viết nhận xét: AI có gợi ý đúng với sở thích của bạn không?
\end{enumerate}

\newpage

% PHẦN 3
\section{Phần 3: Hiểu cách AI học}
\textbf{Thời lượng:} 30 phút

\subsection{Mục tiêu}
Hiểu cơ bản về quá trình học của AI từ input đến output.

\subsection{Nội dung giảng dạy}

\subsubsection{1. Quy trình AI học (15 phút)}
\begin{itemize}[leftmargin=*]
    \item \textbf{Input (Đầu vào):}
    \begin{itemize}
        \item Sở thích của bạn: Ngày, giờ, phòng
        \item Dữ liệu lớp học: Tất cả lớp cần xếp
        \item Lịch sử: Các lớp bạn đã chọn trước đó (nếu có)
    \end{itemize}
    \item \textbf{Xử lý (Processing):}
    \begin{itemize}
        \item AI so sánh từng lớp với sở thích của bạn
        \item Tính điểm phù hợp cho từng lớp
        \item Sử dụng Random Forest để dự đoán chính xác
    \end{itemize}
    \item \textbf{Output (Đầu ra):}
    \begin{itemize}
        \item Điểm số cho mỗi lớp (\texttt{ai\_score})
        \item Danh sách lớp được sắp xếp theo điểm (cao $\rightarrow$ thấp)
        \item Top gợi ý: Lớp nào bạn nên chọn?
    \end{itemize}
\end{itemize}

\subsubsection{2. Ví dụ cụ thể tính điểm (10 phút)}
\begin{itemize}[leftmargin=*]
    \item \textbf{Thông tin lớp:}
    \begin{itemize}
        \item Lớp: \texttt{EE2001-01}
        \item Thứ: Thứ 2
        \item Giờ: 7h-9h
        \item Phòng: D5-401
        \item Giáo viên: Nguyễn Văn A
    \end{itemize}
    \item \textbf{Sở thích của bạn:}
    \begin{itemize}
        \item Ngày ưa thích: Thứ 2, 4, 6
        \item Khung giờ: 7h-11h
        \item Phòng ưa thích: D5-401, C7-205
    \end{itemize}
    \item \textbf{Cách tính điểm:}
    \begin{itemize}
        \item Thứ 2 $\in$ \{Thứ 2, 4, 6\} $\rightarrow$ +1 điểm
        \item 7h-9h $\subset$ 7h-11h $\rightarrow$ +1 điểm
        \item D5-401 $\in$ \{D5-401, C7-205\} $\rightarrow$ +1 điểm
        \item \textbf{Tổng điểm: 3 điểm}
    \end{itemize}
    \item \textbf{So sánh với lớp khác:}
    \begin{itemize}
        \item Lớp B: Thứ 3, 13h-15h, phòng C7-205
        \item Thứ 3 $\notin$ \{Thứ 2, 4, 6\} $\rightarrow$ 0 điểm
        \item 13h-15h $\not\subset$ 7h-11h $\rightarrow$ 0 điểm
        \item C7-205 $\in$ \{D5-401, C7-205\} $\rightarrow$ +1 điểm
        \item \textbf{Tổng điểm: 1 điểm}
    \end{itemize}
    \item \textbf{Kết luận:} Lớp A (3 điểm) được AI đề xuất cao hơn Lớp B (1 điểm)
\end{itemize}

\subsubsection{3. Random Forest - Nhiều cây quyết định (5 phút)}
\begin{itemize}[leftmargin=*]
    \item \textbf{Ý tưởng:} Thay vì 1 người quyết định, có nhiều "chuyên gia" cùng đánh giá
    \item \textbf{Cách hoạt động:}
    \begin{itemize}
        \item Mỗi "cây quyết định" xem xét một khía cạnh khác nhau
        \item Cây 1: Xem xét ngày học
        \item Cây 2: Xem xét khung giờ
        \item Cây 3: Xem xét phòng học
        \item ... (nhiều cây khác)
        \item Kết quả cuối cùng: Trung bình điểm của tất cả các cây
    \end{itemize}
    \item \textbf{Ưu điểm:} Chính xác hơn, ổn định hơn so với chỉ dùng 1 quyết định
\end{itemize}

\subsection{Thực hành nâng cao}
\begin{enumerate}[leftmargin=*]
    \item Thử nghiệm với sở thích rất cụ thể:
    \begin{itemize}
        \item Chỉ thích 1 ngày: \texttt{Mon}
        \item Chỉ thích 1 khung giờ: \texttt{07:00-09:00}
        \item Chỉ thích 1 phòng: \texttt{D5-401}
    \end{itemize}
    \item Chạy lại AI $\rightarrow$ Quan sát điểm AI phân hóa rõ hơn
    \item Giải thích: Sở thích càng cụ thể $\rightarrow$ AI càng chính xác
    \item So sánh với sở thích mở rộng (nhiều ngày, nhiều giờ, nhiều phòng)
\end{enumerate}

\newpage

% TỔNG KẾT
\section{Tổng kết buổi học}

\subsection{Những điểm chính}
\begin{itemize}[leftmargin=*]
    \item AI học từ sở thích của bạn để gợi ý lớp học phù hợp
    \item Mỗi lớp được chấm điểm dựa trên mức độ phù hợp với sở thích
    \item Random Forest là thuật toán AI sử dụng nhiều quyết định để dự đoán chính xác
    \item Sở thích càng cụ thể $\rightarrow$ AI càng chính xác
\end{itemize}

\subsection{Câu hỏi kiểm tra}
\begin{enumerate}[leftmargin=*]
    \item AI là gì? Cho ví dụ về cách AI gợi ý lớp học.
    \item Điểm AI (\texttt{ai\_score}) có ý nghĩa gì? Lớp nào được đề xuất cao nhất?
    \item Tại sao khi thay đổi sở thích, điểm AI của các lớp thay đổi?
    \item Random Forest hoạt động như thế nào? Tại sao nó chính xác hơn?
\end{enumerate}

\subsection{Bài tập về nhà}
\begin{enumerate}[leftmargin=*]
    \item Thử nghiệm với 3 bộ sở thích khác nhau trong \texttt{timetable\_user.csv}
    \item Chạy AI cho mỗi bộ sở thích và ghi lại Top 5 lớp được AI đề xuất
    \item So sánh và giải thích sự khác biệt giữa các kết quả
    \item Viết nhận xét ngắn (100-200 từ): AI có gợi ý đúng với sở thích của bạn không? Cần cải thiện gì?
\end{enumerate}

\subsection{Chuẩn bị cho buổi sau}
\begin{itemize}[leftmargin=*]
    \item Đọc trước nội dung buổi 4: "Tự động xếp thời khóa biểu"
    \item Hiểu các quy tắc xếp lịch: Không trùng giáo viên, không trùng phòng
    \item Xem file \texttt{constraints.json} để hiểu các quy tắc
\end{itemize}

\vspace{1cm}

\begin{center}
    \rule{0.8\textwidth}{0.4pt}\\[0.5cm]
    {\large\bfseries Chúc các em học tốt!}\\[0.3cm]
    {\large Buổi 4: Tự động xếp thời khóa biểu}
\end{center}

\end{document}

